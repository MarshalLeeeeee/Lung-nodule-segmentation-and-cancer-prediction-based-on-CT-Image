\documentclass[hyperref]{article}
%MS%%%%%%%%%%%%%%%%%%%% Article Format %%%%%%%%%%%%%%%%%
%+++++++++++++++++++++ Usepackage +++++++++++++++%%
\usepackage{graphicx} %% Package for Figure
\usepackage{float} %% Package for Float
\usepackage{amssymb}
\usepackage{amsmath}
\usepackage{mathtools}
\usepackage{indentfirst}
\usepackage[thmmarks,amsmath]{ntheorem} %% If amsmath is applied, then amsma is necessary
\usepackage{bm} %% Bold Mathematical Symbols
\usepackage[colorlinks,linkcolor=cyan,citecolor=cyan]{hyperref}
\usepackage{extarrows}
\usepackage[hang,flushmargin]{footmisc} %% Let the footnote not indentation
\usepackage[square,comma,sort&compress,numbers]{natbib} %% Sort of References
\usepackage{mathrsfs} %% Swash letter
\usepackage[font=footnotesize,skip=0pt,textfont=rm,labelfont=rm]{caption,subcaption} 
%% Format of Caption for Tab. and Fig.
\usepackage{booktabs} %% tables with three lines
\usepackage{tocloft}
%+++++++++++++++ Proof etc. +++++++++++++++++++++++++%%
{%% Environment of Proof
    \theoremstyle{nonumberplain}
    \theoremheaderfont{\bfseries}
    \theorembodyfont{\normalfont}
    \theoremsymbol{\mbox{$\Box$}}
    \newtheorem{proof}{Proof}
}

\usepackage{theorem}
\newtheorem{theorem}{Theorem}[section]
\newtheorem{lemma}{Lemma}[section]
\newtheorem{definition}{Definition}[section]
\newtheorem{assumption}{Assumption}[section]
\newtheorem{example}{Example}[section]
\newtheorem{corollary}{Corollary}[section]
{%% Environment of Remark
    \theoremheaderfont{\bfseries}
    \theorembodyfont{\normalfont}
    \newtheorem{remark}{Remark}[section]
}
%\numberwithin{equation}{section} %% Number of Equation
%++++++++++++++++++++++++++++++++ Page format ++++++++++++++++++++++++++%%
\graphicspath{{figure/}}                                 %% Path of Figures
\usepackage[a4paper]{geometry}                           %% Paper size
\geometry{left=2.5cm,right=2.5cm,top=2.5cm,bottom=2.5cm} %% Margin
\linespread{1.2}                                         %% Line Spread
%MS%%%%%%%%%%%%%%%%%%%%%%%%%%%% End Format %%%%%%%%%%%%%%%%%%%%%%%%%%%%%%%%%%

%MS%%%%%%%%%%%%%%%%%%%%%%%%%%%%%%%%%%%%%%%%%%%
%MS                                         %%
%MS        The Main Body begins here        %%
%MS                                         %%
%MS%%%%%%%%%%%%%%%%%%%%%%%%%%%%%%%%%%%%%%%%%%%

%MS+++++++++++++++++++++++ Title +++++++++++++++++++
\begin{document}
\title{\bf Virtual Reality - Final Project Proposal Template} 
\author{\textit{Lung nodule segmentation and cancer prediction based on CT Image}}
%\renewcommand{\thefootnote}{\fnsymbol{footnote}}
%\footnotetext[1]{Corresponding author. }
\maketitle

%MS+++++++++++++++++++++ Member List +++++++++++++++++++++++++
\section{Member List}
{\noindent \large{Li Minchao \hspace{1.65cm}515030910361}\\}
{\noindent \large{\hspace{1cm} Zhang Yiheng \hspace{1.03cm} 515030910216}\\}
{\noindent \large{\hspace{1cm} Wang Yiqing  \hspace{1.2cm} 515030910456}}
%MS++++++++++++++++++++++++++++++ Main body ++++++++++++++++++++
\section{Problem Statement}
We will develop a computer-aided detection and diagnosis system to implement lung nodule segmentation and cancer prediction based on CT images. A deep neural network with CT images as the input will be developed and trained by us to tell if patients have some benign and malignant nodules and segment the malignant parts out.

\section{Problem Description}
With the depletion of the demographic bonus, China is facing the threats of deficit of labor force, including medical care. To counter the deficiency of the doctors, computer-aided diagnosis system may help. Thus, we aim to design a lung nodule segmentation and cancer prediction model based on CT images using deep learning, which may also be helpful for other image based diagnosis system. We will collect data and training our network referring to some state-of-art techniques and network. We hope we can achieve higher accuracy or higher efficiency.

\section{Project Goals and Objectives/Deliverables}
\begin{itemize}
	\item Better results compared to some state-of-art networks in image segmentation, especially for medical use
	\item Combined network features to accomplish multiple tasks at one time
\end{itemize}

\section{Project Scope}
Our focus is optimizing the structure and parameters of the model to achieve better performance and we will not improve the performance on hardware layer, like developing our model on some exclusive hardware platform or developing some advanced pipeline technique.

\section{Success Factors and Benefits}
In the application layer, this project is aimed to improve the efficiency of the process of lung nodule diagnosis and segmentation, and of cancer prediction, which will alleviate the burden of doctors. 

In the technique layer, this project will introduce a framework for image based medical analysis for the future development.

\section{Timeline}
\begin{center}
	\begin{tabular}{c|c}
		Month & Goal \\
		\hline
		March & collect the training data\\
		\hline
		April & make the framework of the net\\
		\hline
		May   & improve the accuracy of the detection\\
		\hline
		<<<<<<< HEAD
		June  & improve the segmentation anf overall performance\\
		\hline
	\end{tabular}
\end{center}

\section{Assumptions}
\subsection{Data Scale}
The number of labeled data collected may be restricted. We may probably to enlarge our dataset by some techniques like mirror, noise or some other methods. Still, we can label it manually.
\subsection{Hardware}
Some deep learning network relies on the performance of powerful GPU. If so, we may run our network on some servers.

\section{Limitations/Restrictions}
Lung analysis techniques have been improved over the last decade. However, there still are issues to be solved such as developing new and better techniques of contrast enhancement and selecting a appropriate criteria for performance evaluation is also critical. 

\end{document}